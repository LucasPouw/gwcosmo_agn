\documentclass[a4paper,10pt]{article}

\usepackage[margin=1.0in]{geometry}
\usepackage{amsmath}
\usepackage{graphicx}
\usepackage{todonotes}

\newcommand\note[1]{\textcolor{red}{#1}}

\title{A summary of the maths in the gwcosmo pipeline}
\author{Rachel Gray}
\date{\today} % delete this line to display the current date

%%% BEGIN DOCUMENT
\begin{document}

\maketitle
%\tableofcontents

\section{Introduction}
Outlined in this document is the maths which has gone into the gwcosmo pipeline, designed to calculate the Hubble constant, $H_0$, using gravitational wave (GW) data from binary black holes (BBHs) and binary neutron stars (BNSs), in combination with EM data in the form of galaxy catalogues or EM counterparts.


\section{An Overview of the Method}


The posterior probability on $H_0$ from $N_{det}$ GW events is computed as follows:
\begin{equation}
\begin{aligned}
p(H_0|\{x_{\text{GW}}\},\{D_{\text{GW}}\},I)&=\frac{p(H_0|I)p(N_{det}|H_0)\prod_i^{N_{det}} p({x_{\text{GW}}}_i|{D_{\text{GW}}}_i,H_0,I)}{p(\{x_{\text{GW}}\}|\{D_{\text{GW}}\},I)}
\\ &\propto p(H_0|I)p(N_{det}|H_0)\prod_i^{N_{det}} p({x_{\text{GW}}}_i|{D_{\text{GW}}}_i,H_0,I)
\end{aligned}
\end{equation}
where $\{x_{\text{GW}}\}$ is the set of GW data and $D_{\text{GW}}$ indicates that the event was detected as a GW.

The term $p(N_{det}|H_0)$ is the likelihood of detecting $N_{det}$ events for a particular choice of $H_0$.  It depends on the instrinsic astrophysical rate of events, $R=\frac{\partial{N}}{\partial V \partial T}$, and by choosing a prior on rate of $p(R|I) \propto 1/R$, then the dependence on $H_0$ drops out. For simplicity this approximation is made throughout the analysis.

The remaining term factorises into likelihoods for each detected event,
which can each be written as $p(x_{\text{GW}}|H_0,D_{\text{GW}},I)$.
\begin{equation}
\begin{aligned}
p(x_{\text{GW}}|H_0,D_{\text{GW}},I) &= \dfrac{p(D_{\text{GW}}|x_{\text{GW}},H_0,I)p(x_{\text{GW}}|H_0,I)}{p(D_{\text{GW}}|H_0,I)},
\\ &= \dfrac{p(x_{\text{GW}}|H_0,I)}{p(D_{\text{GW}}|H_0,I)},
\end{aligned} 
\end{equation}
in the case where $x_{\text{GW}}$ passes some SNR threshold and $p(D_{\text{GW}}|x,H_0,I)=1$.

\subsection{The statistical method}
In the statistical case, the EM information enters the analysis as a prior, in the form of a galaxy catalogue, made up of a series of delta functions \footnote{when uncertainties are ignored} on redshift, RA and Dec.  As we are in the regime where (especially for BBHs) galaxy catalogues cannot be considered complete out to the distances to which GW events are detectable, we have to consider the possibility that the host galaxy is not contained within the galaxy catalogue, but lies somewhere beyond it.

In order to do so, we marginalise the likelihood over the case where the host galaxy is, and is not, in the catalogue (denoted by $G$ and $\bar{G}$ respectively):
\begin{equation} \label{Eq:sum G}
\begin{aligned}
p(x_{\text{GW}}|H_0,D_{\text{GW}},I) &= \sum_{g=G,\bar{G}} p(x_{\text{GW}},g|H_0,D_{\text{GW}},I)
\\ &= \sum_{g=G,\bar{G}} p(x_{\text{GW}}|H_0,g,D_{\text{GW}},I) p(g|H_0,D_{\text{GW}},I)
\\ &= p(x_{\text{GW}}|H_0,G,D_{\text{GW}},I) p(G|H_0,D_{\text{GW}},I) + p(x_{\text{GW}}|H_0,\bar{G},D_{\text{GW}},I) p(\bar{G}|H_0,D_{\text{GW}},I)
\end{aligned} 
\end{equation}

This equation,
\begin{equation} \label{Eq:Master Equation}
\begin{aligned}
p(H_0|x_{\text{GW}},D_{\text{GW}},I) &= \dfrac{p(x_{\text{GW}}|H_0,G,I)}{p(D_{\text{GW}}|H_0,G,I)} p(G|H_0,D_{\text{GW}},I) + \dfrac{p(x_{\text{GW}}|H_0,\bar{G},I)}{p(D_{\text{GW}}|H_0,\bar{G},I)} p(\bar{G}|H_0,D_{\text{GW}},I),
\end{aligned}
\end{equation}
is what has come to be known as the ``Master Equation'' (desperately in need of a new name), and calculating the six constituent parts forms the bulk of the gwcosmo analysis.


\subsection{The EM counterpart method}
The method outlined above is for the statistical $H_0$ case, in which no EM counterpart is observed, or expected (eg. for BBHs).  We now consider the case where we expect to observe an EM counterpart (eg. BNSs).  The main difference this change implies is the inclusion of a likelihood term for the EM counterpart data, \emph{in addition} to the galaxy catalogue already in use.

Similar to above, we can express the likelihood in this case as follows:
\begin{equation}
\begin{aligned}
p(H_0|x_{\text{GW}},x_{\text{EM}},D_{\text{GW}},D_{\text{EM}},C,I) \propto p(x_{\text{GW}},x_{\text{EM}}|H_0,D_{\text{GW}},D_{\text{EM}},C,I) p(H_0|D_{\text{GW}},D_{\text{EM}},C)
\end{aligned} 
\end{equation}
where $C$ refers to the EM catalogue, and other symbols have their usual meanings.

First we marginalise over the cases where the source is inside and outside the galaxy catalogue:
\begin{equation}
\begin{aligned}
p(x_{\text{GW}},x_{\text{EM}}|H_0,D_{\text{GW}},D_{\text{EM}},C,I) &= \sum_{g=G,\bar{G}} p(x_{\text{GW}},x_{\text{EM}},g|H_0,D_{\text{GW}},D_{\text{EM}},C,I)
\\ &= \sum_{g=G,\bar{G}} p(x_{\text{GW}},x_{\text{EM}}|H_0,g,D_{\text{GW}},D_{\text{EM}},C,I) p(g|H_0,D_{\text{GW}},D_{\text{EM}},C,I)
\end{aligned} 
\end{equation}

And then we Bayes the first expression:
\begin{equation}
\begin{aligned}
p(x_{\text{GW}},x_{\text{EM}}|H_0,g,D_{\text{GW}},D_{\text{EM}},C,I) &= \dfrac{p(x_{\text{GW}},x_{\text{EM}}|H_0,g,C,I) p(D_{\text{GW}},D_{\text{EM}}|x_{\text{GW}},x_{\text{EM}},H_0,g,C,I)}{p(D_{\text{GW}},D_{\text{EM}}|H_0,g,C,I)}
\\&= \dfrac{p(x_{\text{GW}},x_{\text{EM}}|H_0,g,C,I)}{p(D_{\text{GW}},D_{\text{EM}}|H_0,g,C,I)} 
\end{aligned} 
\end{equation}
We take $p(D_{\text{GW}},D_{\text{EM}}|x_{\text{GW}},x_{\text{EM}},H_0,g,C,I)=1$ whenever we have GW and EM data (note this currently doesn't include the case in which a GW BNS was detected, and no counterpart was detected).

Both the numerator and denominator can now be expanded:
\begin{equation}
\begin{aligned}
p(x_{\text{GW}},x_{\text{EM}}|H_0,g,D_{\text{GW}},D_{\text{EM}},C,I) &= \dfrac{p(x_{\text{GW}}|H_0,g,C,I) p(x_{\text{EM}}|H_0,g,C,I)}{p(D_{\text{EM}}|D_{\text{GW}},H_0,g,C,I) p(D_{\text{GW}}|H_0,g,C,I)} 
\\ &= \dfrac{p(x_{\text{GW}}|H_0,g,C,I) p(x_{\text{EM}}|H_0,g,C,I)}{p(D_{\text{GW}}|H_0,g,C,I)} 
\end{aligned} 
\end{equation}
where we take $p(D_{\text{EM}}|D_{\text{GW}},H_0,g,C,I) = 1$, under the assumption that if the event was detected in gravitational waves, it will be detectable to EM telescopes (which is true if the telescope observing it sees beyond the BNS horizon, as is currently the case).  Note that this still involves many assumptions, such as assuming that the EM counterpart is not obscured behind another galaxy, or the galactic plane, and these are assumptions which will have to be dealt with in the future.


\subsubsection{The complete catalogue case}
For the sake of simplicity, completeness can be assumed for all BNSs with EM counterparts which have already been observed (again, something that will not necessarily be the case in the future).  In this case the likelihood for the event simplifies down to the following:
\begin{equation}
\begin{aligned}
p(x_{\text{GW}},x_{\text{EM}}|H_0,D_{\text{GW}},D_{\text{EM}},C,I) &= \dfrac{p(x_{\text{GW}}|H_0,G,C,I) p(x_{\text{EM}}|H_0,G,C,I)}{p(D_{\text{GW}}|H_0,G,C,I)} 
\end{aligned} 
\end{equation}
where $G$ now indicates that the source host is inside the galaxy catalogue.




\subsection{A brief note on luminosity weighting}
Included in the term $I$ on the right-hand side of all of these equations is one very important assumption: that there really was a GW source (here we shall denote it $s$), and the detection was not a false alarm.  In the majority of calculations, the presence of this term can be taken as granted and safely ignored, as it has no bearing on the result.  For example:
\begin{equation}
\begin{aligned}
p(z|s,I) &= \dfrac{p(s|z,I) p(z|I)}{p(s|I)} 
\\ &= \dfrac{p(s|I) p(z|I)}{p(s|I)} 
\\ &= p(z|I),
\end{aligned}
\end{equation}
that is, we do not modify our prior on $z$ because of our knowledge.

However, there is an important case in which this assumption is true only part of the time:
\begin{equation}
\begin{aligned}
p(M|s,H_0,I) &= \dfrac{p(s|M,H_0,I)p(M|H_0,I)}{p(s|H_0,I)} 
\\ &= p(M|H_0,I)
\end{aligned}
\end{equation}
which is \emph{only} true if we believe that the probability of a particular galaxy being host to a gravitational wave event is independent of the galaxy's absolute magnitude. That is, when there is no luminosity weighting.

In general, we define $p(M|H_0,I)$ as a distribution of absolute magnitudes represented by the Schechter function, as we believe this mirrors the distribution of absolute magnitudes for all the galaxies in the universe.  If we believe that the probability of a given galaxy being host to the gravitational wave source is dependent on the galaxy's absolute magnitude, then $p(s|M,H_0,I)$ takes some non-constant value.  For example:
\begin{equation}
\begin{aligned}
p(s|M,H_0,I) &\propto L(M)
\end{aligned}
\end{equation}

As will be seen below, all $p(s|H_0,I)$ terms will cancel out in the numerator and denominator, and therefore can be safely ignored.






\section{Individual components}
Now to consider the individual components of Eq. \ref{Eq:Master Equation}.


\subsection{Likelihood when host is in catalogue: $p(x_{\text{GW}}|H_0,G,D_{\text{GW}},I)$}


Marginalising over redshift, sky location, apparent magnitude and absolute magnitude:
\begin{equation}
\begin{aligned}
p(x_{\text{GW}}|H_0,G,D_{\text{GW}},I) &= \dfrac{\iiiint p(x_{\text{GW}}|H_0,G,z,\Omega,m,M,I) p(z,\Omega,m,M|s,H_0,G,I) dz d\Omega dm dM}{\iiiint p(D_{\text{GW}}|H_0,G,z,\Omega,m,M,I) p(z,\Omega,m,M|s,H_0,G,I) dz d\Omega dm dM}
\\ &= \dfrac{\iiiint p(x_{\text{GW}}|H_0,z,\Omega,I) p(z,\Omega,m,M|s,H_0,G,I) dz d\Omega dm dM}{\iiiint p(D_{\text{GW}}|H_0,z,\Omega,I) p(z,\Omega,m,M|s,H_0,G,I) dz d\Omega dm dM}
\end{aligned}
\end{equation}
which is true \emph{if} we can assume that both $x_{\text{GW}}$ and $D_{\text{GW}}$ are independent of $G$, $m$ and $M$.  Here, the dependence of the priors on $G$ simply means that we take the prior to be the galaxies within the catalogue, as a series of delta functions with specific $z$, $\Omega$ and $m$ values.

As the priors on $z$, $\Omega$, $m$ and $M$ are connected through the specific galaxies inside the catalogue, expanding must be done with care:
\begin{equation}
\begin{aligned}
p(z,\Omega,m,M|s,H_0,G,I) &= p(M|s,H_0,z,\Omega,m,G,I)p(z,\Omega,m|s,H_0,G,I)
\\ &= \dfrac{p(s|M,H_0,z,\Omega,m,G,I) p(M|H_0,z,\Omega,m,G,I)}{p(s|H_0,z,\Omega,m,G,I)} \dfrac{p(s|H_0,z,\Omega,m,G,I) p(z,\Omega,m|H_0,G,I)}{p(s|H_0,G,I)} 
\\ &= \dfrac{p(s|M,I) \delta(M-M(H_0,z,m)) p(z,\Omega,m|G,I)}{p(s|H_0,G,I)}
\end{aligned}
\end{equation}


The form taken by $p(s|M(H_0,z,m),I)$ depends on how we believe a galaxy's absolute magnitude is related to its probability of being a GW host.
\begin{equation}
\begin{aligned}
p(s|M(H_0,z,m),I) &\propto 
\begin{cases}
L(M(H_0,z,m)) & \text{if luminosity weighted}\\
\text{const} & \text{if unweighted}
\end{cases}
\end{aligned}
\end{equation}

\begin{equation}
\begin{aligned}
p(x_{\text{GW}}|H_0,G,D_{\text{GW}},I) &= \dfrac{p(s|H_0,G,I)}{p(s|H_0,G,I)} \dfrac{\iiint p(x_{\text{GW}}|H_0,z,\Omega,I) p(s|M(H_0,z,m),I) p(z,\Omega,m|G,I) dz d\Omega dm}{\iiint p(D_{\text{GW}}|H_0,z,\Omega,I) p(s|M(H_0,z,m),I) p(z,\Omega,m|G,I) dz d\Omega dm}
\\ &= \dfrac{\iiint p(x_{\text{GW}}|H_0,z,\Omega,I) p(s|M(H_0,z,m),I) p(z,\Omega,m|G,I) dz d\Omega dm}{\iiint p(D_{\text{GW}}|H_0,z,\Omega,I) p(s|M(H_0,z,m),I) p(z,\Omega,m|G,I) dz d\Omega dm}
\\ &= \dfrac{\sum^N_{i=1} p(x_{\text{GW}}|H_0,z_i,\Omega_i,I) p(s|M(H_0,z_i,m_i),I)}{\sum^N_{i=1} p(D_{\text{GW}}|H_0,z_i,\Omega_i,I) p(s|M(H_0,z_i,m_i),I)}
\end{aligned}
\end{equation}

In the unweighted case, this simplifies to the following:
\begin{equation}
\begin{aligned}
p(x_{\text{GW}}|H_0,G,D_{\text{GW}},I) &= \dfrac{\sum^N_{i=1} p(x_{\text{GW}}|H_0,z_i,\Omega_i,I) }{\sum^N_{i=1} p(D_{\text{GW}}|H_0,z_i,\Omega_i,I)}
\end{aligned}
\end{equation}









\subsection{Probability the host galaxy is in the galaxy catalogue: $p(G|H_0,D_{\text{GW}},I)$}


\begin{equation}
\begin{aligned}
p(G|H_0,D_{\text{GW}},I) &= \iiiint p(G,z,\Omega,m,M|H_0,D_{\text{GW}},I) dz d\Omega dm dM
\\ &= \iiiint p(G|H_0,D_{\text{GW}},z,\Omega,m,M,I) p(z,\Omega,m,M|H_0,D_{\text{GW}},I) dz d\Omega dm dM
\\ &= \iiiint \Theta[m_{\text{th}}-m] \dfrac{p(D_{\text{GW}}|H_0,z,\Omega,m,M,I) p(z,\Omega,m,M|H_0,I)}{p(D_{\text{GW}}|H_0,I)}  dz d\Omega dm dM
\\ &=  \dfrac{\iiiint \Theta[m_{\text{th}}-m] p(D_{\text{GW}}|H_0,z,\Omega,m,M,I) p(z,\Omega,m,M|H_0,I) dz d\Omega dm dM}{\iiiint p(D_{\text{GW}},z,\Omega,m,M|H_0,I) dz d\Omega dm dM} 
\\ &=  \dfrac{\iiiint \Theta[m_{\text{th}}-m] p(D_{\text{GW}}|H_0,z,\Omega,I) p(z,\Omega,m,M|H_0,I) dz d\Omega dm dM}{\iiiint p(D_{\text{GW}}|H_0,z,\Omega,I) p(z,\Omega,m,M|H_0,I) dz d\Omega dm dM} 
\end{aligned}
\end{equation}
where $p(G|H_0,D_{\text{GW}},z,\Omega,m,M,I)$ is a heaviside step function around $m = m_{\text{th}}$ - the apparent magnitude threshold of the galaxy catalogue.

\begin{equation}
\begin{aligned}
p(G|H_0,D_{\text{GW}},I) &=  \dfrac{\iiiint \Theta[m_{\text{th}}-m] p(D_{\text{GW}}|H_0,z,\Omega,I) p(z,\Omega,m,M|H_0,I) dz d\Omega dm dM}{\iiiint p(D_{\text{GW}}|H_0,z,\Omega,I) p(z,\Omega,m,M|H_0,I) dz d\Omega dm dM} 
\\ &= \dfrac{\iiiint \Theta[m_{\text{th}}-m] p(D_{\text{GW}}|H_0,z,\Omega,I) p(z)p(\Omega)p(M|H_0,I)\delta(m - m(z,H_0,M)) dz d\Omega dm dM}{\iiiint p(D_{\text{GW}}|H_0,z,\Omega,I) p(z)p(\Omega)p(M|H_0,I)\delta(m - m(z,H_0,M)) dz d\Omega dm dM}
\\ &= \dfrac{\iiint \Theta[m_{\text{th}}-m(z,H_0,M)] p(D_{\text{GW}}|H_0,z,\Omega,I) p(z)p(\Omega)p(M|H_0,I)dz d\Omega dM}{\iiint p(D_{\text{GW}}|H_0,z,\Omega,I) p(z)p(\Omega)p(M|H_0,I) dz d\Omega dM}
\\ &= \dfrac{\int^{z(H_0,m_{\text{th}},M)}_0 dz \int d\Omega \int dM p(D_{\text{GW}}|H_0,z,\Omega,I) p(z)p(\Omega)p(M|H_0,I)}{\iiint p(D_{\text{GW}}|H_0,z,\Omega,I) p(z)p(\Omega)p(M|H_0,I) dz d\Omega dM}
\end{aligned}
\end{equation}


\subsubsection{The luminosity weighted case}
In the case where we consider luminosity weighting, this takes the following form:
\begin{equation}
\begin{aligned}
p(G|H_0,D_{\text{GW}},I) &= \dfrac{\int^{z(H_0,m_{\text{th}},M)}_0 dz \int d\Omega \int dM p(D_{\text{GW}}|H_0,z,\Omega,I) p(z)p(\Omega) p(s|M,I) p(M|H_0,I)}{\iiint p(D_{\text{GW}}|H_0,z,\Omega,I) p(z)p(\Omega) p(s|M,I) p(M|H_0,I) dz d\Omega dM}
\end{aligned}
\end{equation}


\subsection{Probability the host galaxy is not in the galaxy catalogue: $p(\bar{G}|H_0,D_{\text{GW}},I)$}

As the probabilities of being in the catalogue and not in the catalogue must add up the one, we can calculate the counterpart to $p(G|H_0,D_{\text{GW}},I)$ as follows:
\begin{equation}
\begin{aligned}
p(\bar{G}|H_0,D_{\text{GW}},I) &= 1 - p(G|H_0,D_{\text{GW}},I)
\end{aligned}
\end{equation}








\subsection{Likelihood when host is not in catalogue: $p(x_{\text{GW}}|H_0,\bar{G},D_{\text{GW}},I)$}
Similarly:
\begin{equation}
\begin{aligned}
p(x_{\text{GW}}|H_0,\bar{G},D_{\text{GW}},I) &= \dfrac{\iiiint p(x_{\text{GW}}|H_0,\bar{G},z,\Omega,I) p(z,\Omega,m,M|H_0,\bar{G},I) dz d\Omega dm dM}{\iiiint p(D_{\text{GW}}|H_0,\bar{G},z,\Omega,I) p(z,\Omega,m,M|H_0,\bar{G},I) dz d\Omega dm dM}
\\ &= \dfrac{\iiiint p(x_{\text{GW}}|H_0,z,\Omega,I) p(z,\Omega,m,M|H_0,\bar{G},I) dz d\Omega dm dM}{\iiiint p(D_{\text{GW}}|H_0,z,\Omega,I) p(z,\Omega,m,M|H_0,\bar{G},I) dz d\Omega dm dM}
\\ &= \dfrac{\iiiint p(x_{\text{GW}}|H_0,z,\Omega,I) \dfrac{p(\bar{G}|H_0,z,\Omega,m,M,I)p(z,\Omega,m,M|H_0,I)}{p(\bar{G}|H_0,I)} dz d\Omega dm dM}{\iiiint p(D_{\text{GW}}|H_0,z,\Omega,I) \dfrac{p(\bar{G}|H_0,z,\Omega,m,M,I)p(z,\Omega,m,M|H_0,I)}{p(\bar{G}|H_0,I)} dz d\Omega dm dM}
\\ &= \dfrac{p(\bar{G}|H_0,I)}{p(\bar{G}|H_0,I)}\dfrac{\iiiint p(x_{\text{GW}}|H_0,z,\Omega,I) p(\bar{G}|m,I)p(z,\Omega,m,M|H_0,I) dz d\Omega dm dM}{\iiiint p(D_{\text{GW}}|H_0,z,\Omega,I) p(\bar{G}|m,I)p(z,\Omega,m,M|H_0,I) dz d\Omega dm dM}
\\ &= \dfrac{\iiiint p(x_{\text{GW}}|H_0,z,\Omega,I) p(\bar{G}|m,I)p(z,\Omega,m,M|H_0,I) dz d\Omega dm dM}{\iiiint p(D_{\text{GW}}|H_0,z,\Omega,I) p(\bar{G}|m,I)p(z,\Omega,m,M|H_0,I) dz d\Omega dm dM}
\end{aligned}
\end{equation}
which is true \emph{if} we can assume that both $x_{\text{GW}}$ and $D_{\text{GW}}$ are mutually independent of $G$.

\begin{equation}
\begin{aligned}
p(x_{\text{GW}}|H_0,\bar{G},D_{\text{GW}},I) &= \dfrac{\iiiint p(x_{\text{GW}}|H_0,z,\Omega,I) \Theta(m-m_{\text{th}})p(z)p(\Omega)\delta(m-m(z,H_0,M))p(M|H_0,I) dz d\Omega dm dM}{\iiiint p(D_{\text{GW}}|H_0,z,\Omega,I) \Theta(m-m_{\text{th}})p(z)p(\Omega)\delta(m-m(z,H_0,M))p(M|H_0,I) dz d\Omega dm dM}
\\ &= \dfrac{\iiint p(x_{\text{GW}}|H_0,z,\Omega,I) \Theta(m(z,H_0,M)-m_{\text{th}}) p(z)p(\Omega)p(M|H_0,I) dz d\Omega dM}{\iiint p(D_{\text{GW}}|H_0,z,\Omega,I) \Theta(m(z,H_0,M)-m_{\text{th}})p(z)p(\Omega)p(M|H_0,I) dz d\Omega dM}
\\ &= \dfrac{\int^\infty_{z(H_0,m_{\text{th}},M)} \iint p(x_{\text{GW}}|H_0,z,\Omega,I) p(z)p(\Omega)p(M|H_0,I) dz d\Omega dM}{\int^\infty_{z(H_0,m_{\text{th}},M)} \iint p(D_{\text{GW}}|H_0,z,\Omega,I) p(z)p(\Omega)p(M|H_0,I) dz d\Omega dM}
\end{aligned}
\end{equation}


\subsubsection{The luminosity weighted case}
In the case where we consider luminosity weighting, this takes the following form:
\begin{equation}
\begin{aligned}
p(x_{\text{GW}}|H_0,\bar{G},D_{\text{GW}},I) &= \dfrac{\int^\infty_{z(H_0,m_{\text{th}},M)} \iint p(x_{\text{GW}}|H_0,z,\Omega,I) p(z) p(\Omega) p(s|M,I) p(M|H_0,I) dz d\Omega dM}{\int^\infty_{z(H_0,m_{\text{th}},M)} \iint p(D_{\text{GW}}|H_0,z,\Omega,I) p(z) p(\Omega) p(s|M,I) p(M|H_0,I) dz d\Omega dM}
\end{aligned}
\end{equation}





\section{Converting from maths to code}
There are several things to bear in mind when moving from the maths shown above, to actually doing the calculations in python.

\subsection{$p(G|H_0,D_{\text{GW}},I)$ and $p(\bar{G}|H_0,D_{\text{GW}},I)$}
Instead of calculating 
\begin{equation}
\begin{aligned}
p(G|H_0,D_{\text{GW}},I) &= \dfrac{\int^{z(H_0,m_{\text{th}},M)}_0 dz \int d\Omega \int dM p(D_{\text{GW}}|H_0,z,\Omega,I) p(z)p(\Omega)p(M|H_0,I)}{\iiint p(D_{\text{GW}}|H_0,z,\Omega,I) p(z)p(\Omega)p(M|H_0,I) dz d\Omega dM},
\end{aligned}
\end{equation}
we can save time by reducing the number of integrals, by noting that this is the same as the following:
\begin{equation}
\begin{aligned}
p(G|H_0,D_{\text{GW}},I) &= \dfrac{\int^{z(H_0,m_{\text{th}},M)}_0 dz \int dM p(D_{\text{GW}}|H_0,z,I) p(z)p(M|H_0,I)}{\iint p(D_{\text{GW}}|H_0,z,I) p(z)p(M|H_0,I) dz dM}
\end{aligned}
\end{equation}
where the factors of $4\pi$ that come from integrating over the entire sky cancel in numerator and denominator.




\subsection{KDEs}
When we create a KDE on the event samples, what we make is $p(d_L,\Omega|x_{\text{GW}})$, not $p(x_{\text{GW}}|d_L,\Omega)$.

Also, we first use a 2+1D KDE approach, where we assume that $d_L$ and $\Omega$ are mutually independent (untrue), in order to split the KDE up as follows:
\begin{equation}
\begin{aligned}
p(d_L,\Omega|x_{\text{GW}}) &= p(d_L|x_{\text{GW}}) p(\Omega|x_{\text{GW}})
\\ &= \dfrac{p(x_{\text{GW}}|d_L) p(d_L)}{p(x_{\text{GW}})} \dfrac{p(x_{\text{GW}}|\Omega) p(\Omega)}{p(x_{\text{GW}})}
\end{aligned}
\end{equation}

Essentially, when we create the KDE it already contains the prior that was used to generate the samples (ie $d_L^2$ or $1/4\pi$).  This means that if we want to change the prior we are using, we need to first remove the one that is already present.  In particular, this means being careful when summing over the galaxies in the catalogue, as they represent a new prior on $\alpha,\delta,z$, and so the old priors need to be removed in this case.







\section{For the future: the pixel-based method}
In order to take into account the fact that galaxy catalogues have varying levels of completeness across the sky, we consider a method in which the sky is gridded up into equally-sized pieces, which are later summed.

\begin{equation}
\begin{aligned}
p(x_{\text{GW}}|H_0,D_{\text{GW}},I) &= \int p(x_{\text{GW}},\Omega|H_0,D_{\text{GW}},I) d\Omega
\\ & = \int p(x_{\text{GW}}|H_0,D_{\text{GW}},\Omega,I) p(\Omega|H_0,D_{\text{GW}},I) d\Omega
\\ & = \int p(x_{\text{GW}}|H_0,D_{\text{GW}},\Omega,I) \dfrac{p(D_{\text{GW}}|H_0,\Omega,I)p(\Omega|H_0,I)}{p(D_{\text{GW}}|H_0,I)}  d\Omega
\\ &= \dfrac{1}{p(D_{\text{GW}}|H_0,I)} \int p(x_{\text{GW}}|H_0,D_{\text{GW}},\Omega,I) p(D_{\text{GW}}|H_0,\Omega,I)p(\Omega|I) d\Omega
\\ &= \dfrac{1}{p(D_{\text{GW}}|H_0,I)} \sum^{N_{\text{pix}}}_i \bigg[p(x_{\text{GW}}|H_0,D_{\text{GW}},\Omega_i,I) p(D_{\text{GW}}|H_0,\Omega_i,I)p(\Omega_i|I)\bigg]
\end{aligned} 
\end{equation}

Looking specifically at $p(x_{\text{GW}}|H_0,D_{\text{GW}},\Omega_i,I)$ and expanding as in previous sections:
\begin{equation}
\begin{aligned}
p(x_{\text{GW}}|H_0,D_{\text{GW}},\Omega_i,I) &= \dfrac{p(x_{\text{GW}}|H_0,G,\Omega_i,I)}{p(D_{\text{GW}}|H_0,G,\Omega_i,I)} p(G|H_0,D_{\text{GW}},\Omega_i,I) + \dfrac{p(x_{\text{GW}}|H_0,\bar{G},\Omega_i,I)}{p(D_{\text{GW}}|H_0,\bar{G},\Omega_i,I)} p(\bar{G}|H_0,D_{\text{GW}},\Omega_i,I),
\end{aligned}
\end{equation}
and so the final expression becomes:
\begin{equation}
\begin{aligned}
p(x_{\text{GW}}|H_0,D_{\text{GW}},I) = \dfrac{1}{p(D_{\text{GW}}|H_0,I)} &\sum^{N_{\text{pix}}}_i \Bigg[ \bigg( \dfrac{p(x_{\text{GW}}|H_0,G,\Omega_i,I)}{p(D_{\text{GW}}|H_0,G,\Omega_i,I)} p(G|H_0,D_{\text{GW}},\Omega_i,I) \\ &+ \dfrac{p(x_{\text{GW}}|H_0,\bar{G},\Omega_i,I)}{p(D_{\text{GW}}|H_0,\bar{G},\Omega_i,I)} p(\bar{G}|H_0,D_{\text{GW}},\Omega_i,I) \bigg) \times p(D_{\text{GW}}|H_0,\Omega_i,I)p(\Omega_i|I) \Bigg]
\end{aligned} 
\end{equation}

It is also worth noting that, for a suitably fine grid, the choice of 3D vs 2+1D for dealing with GW data is removed, as for every position in the sky a single distance posterior is used, and so this method is inherently ``3D''.





\end{document}