\documentclass[a4paper,10pt]{article}

\usepackage[margin=1.0in]{geometry}
\usepackage{amsmath}
\usepackage{graphicx}
\usepackage{todonotes}

\newcommand\note[1]{\textcolor{red}{#1}}

\title{A summary of the maths in the gwcosmo pipeline}
\author{Rachel Gray and the LVC cosmology group}
\date{\today} % delete this line to display the current date

%%% BEGIN DOCUMENT
\begin{document}

\maketitle
%\tableofcontents

\section{Introduction}
Outlined in this document is the maths which has gone into the gwcosmo pipeline, designed to calculate the Hubble constant, $H_0$, using gravitational wave (GW) data from binary black holes (BBHs) and binary neutron stars (BNSs), in combination with EM data in the form of galaxy catalogues or EM counterparts.

Section \ref{Sec: Overview} introduces the Bayesian approach for the statistical and counterpart cases.  Section \ref{Sec: Components} takes a closer look at the individual components which go into the statistical case, and derives mathematical expressions for each of them.  Section \ref{Sec: maths2code} discusses some adjustments and approximations which have been made to the maths in section \ref{Sec: Components} in order to allow the approach to be efficiently adapted into a pipeline.  And section \ref{Sec: Future} outlines a method which will be implemented during O3, in order to improve upon current approximations.


\section{An Overview of the Method \label{Sec: Overview}}


The posterior probability on $H_0$ from $N_{det}$ GW events is computed as follows:
\begin{equation}
\begin{aligned}
p(H_0|\{x_{\text{GW}}\},\{D_{\text{GW}}\},I)&=\frac{p(H_0|I)p(N_{det}|H_0)\prod_i^{N_{det}} p({x_{\text{GW}}}_i|{D_{\text{GW}}}_i,H_0,I)}{p(\{x_{\text{GW}}\}|\{D_{\text{GW}}\},I)}
\\ &\propto p(H_0|I)p(N_{det}|H_0)\prod_i^{N_{det}} p({x_{\text{GW}}}_i|{D_{\text{GW}}}_i,H_0,I)
\end{aligned}
\end{equation}
where $\{x_{\text{GW}}\}$ is the set of GW data and $D_{\text{GW}}$ indicates that the event was detected as a GW.

The term $p(N_{det}|H_0)$ is the likelihood of detecting $N_{det}$ events for a particular choice of $H_0$.  It depends on the instrinsic astrophysical rate of events, $R=\frac{\partial{N}}{\partial V \partial T}$, and by choosing a prior on rate of $p(R|I) \propto 1/R$, then the dependence on $H_0$ drops out. For simplicity this approximation is made throughout the analysis.

The remaining term factorises into likelihoods for each detected event,
which can each be written as $p(x_{\text{GW}}|H_0,D_{\text{GW}},I)$.
\begin{equation}
\begin{aligned}
p(x_{\text{GW}}|H_0,D_{\text{GW}},I) &= \dfrac{p(D_{\text{GW}}|x_{\text{GW}},H_0,I)p(x_{\text{GW}}|H_0,I)}{p(D_{\text{GW}}|H_0,I)},
\\ &= \dfrac{p(x_{\text{GW}}|H_0,I)}{p(D_{\text{GW}}|H_0,I)},
\end{aligned} 
\end{equation}
in the case where $x_{\text{GW}}$ passes some SNR threshold and $p(D_{\text{GW}}|x,H_0,I)=1$.

\subsection{The statistical method}
In the statistical case, the EM information enters the analysis as a prior, in the form of a galaxy catalogue, made up of a series of delta functions \footnote{when uncertainties are ignored} on redshift, RA and Dec.  As we are in the regime where (especially for BBHs) galaxy catalogues cannot be considered complete out to the distances to which GW events are detectable, we have to consider the possibility that the host galaxy is not contained within the galaxy catalogue, but lies somewhere beyond it.

In order to do so, we marginalise the likelihood over the case where the host galaxy is, and is not, in the catalogue (denoted by $G$ and $\bar{G}$ respectively):
\begin{equation} \label{Eq:sum G}
\begin{aligned}
p(x_{\text{GW}}|H_0,D_{\text{GW}},I) &= \sum_{g=G,\bar{G}} p(x_{\text{GW}},g|H_0,D_{\text{GW}},I)
\\ &= \sum_{g=G,\bar{G}} p(x_{\text{GW}}|H_0,g,D_{\text{GW}},I) p(g|H_0,D_{\text{GW}},I)
\\ &= p(x_{\text{GW}}|H_0,G,D_{\text{GW}},I) p(G|H_0,D_{\text{GW}},I) + p(x_{\text{GW}}|H_0,\bar{G},D_{\text{GW}},I) p(\bar{G}|H_0,D_{\text{GW}},I)
\end{aligned} 
\end{equation}


\subsubsection{The catalogue patch case}
While in general the statistical method in gwcosmo is designed for use with a galaxy catalogue which covers the entire sky, a small modification allows the use of catalogues which only cover a patch of sky, as long as the patch can be specified using limits in RA and Dec.  If we represent the sky area covered by the catalogue as $\Omega_{\text{cat}}$, and the area outside the catalogue as $\Omega_{\text{rest}}$, such that $\Omega_{\text{cat}}+\Omega_{\text{rest}}$ covers the whole sky, this can be written as follows:
\begin{equation}
\begin{aligned}
p(x_{\text{GW}}|H_0,D_{\text{GW}},I) &= \int p(x_{\text{GW}}|H_0,D_{\text{GW}},\Omega,I)p(\Omega|I) d\Omega
\\&=  \int^{\Omega_{\text{cat}}} p(x_{\text{GW}}|H_0,D_{\text{GW}},\Omega,I)p(\Omega|I) d\Omega + \int^{\Omega_{\text{rest}}}p(x_{\text{GW}}|H_0,D_{\text{GW}},\Omega,I)p(\Omega|I) d\Omega.
\end{aligned} 
\end{equation}
The first term is equivalent to Eq. \ref{Eq:sum G} with limits on the integral over $\Omega$, while the second term has no $G$ and $\bar{G}$ terms, and covers the rest of the sky from redshift 0 to $\infty$.


\subsection{The EM counterpart method}
The method outlined above is for the statistical $H_0$ case, in which no EM counterpart is observed, or expected (eg. for BBHs).  We now consider the case where we expect to observe an EM counterpart (eg. BNSs).  The main difference this change implies is the inclusion of a likelihood term for the EM counterpart data, \emph{in addition} to the galaxy catalogue already in use.

Similar to above, we can express the likelihood in this case as follows:
\begin{equation}
\begin{aligned}
p(x_{\text{GW}},x_{\text{EM}}|H_0,D_{\text{GW}},D_{\text{EM}},I) &= \dfrac{p(x_{\text{GW}},x_{\text{EM}}|H_0,I) p(D_{\text{GW}},D_{\text{EM}}|x_{\text{GW}},x_{\text{EM}},H_0,I)}{p(D_{\text{GW}},D_{\text{EM}}|H_0,I)}
\\&= \dfrac{p(x_{\text{GW}},x_{\text{EM}}|H_0,I)}{p(D_{\text{GW}},D_{\text{EM}}|H_0,I)}.
\end{aligned} 
\end{equation}
We take $p(D_{\text{GW}},D_{\text{EM}}|x_{\text{GW}},x_{\text{EM}},H_0,I)=1$ whenever we have GW and EM data.

Both the numerator and denominator can now be expanded:
\begin{equation} \label{Eq:counterpart}
\begin{aligned}
p(x_{\text{GW}},x_{\text{EM}}|H_0,D_{\text{GW}},D_{\text{EM}},I) &= \dfrac{p(x_{\text{GW}}|H_0,I) p(x_{\text{EM}}|H_0,I)}{p(D_{\text{EM}}|D_{\text{GW}},H_0,I) p(D_{\text{GW}}|H_0,I)} 
\\ &= \dfrac{p(x_{\text{GW}}|H_0,I) p(x_{\text{EM}}|H_0,I)}{p(D_{\text{GW}}|H_0,I)} 
\end{aligned} 
\end{equation}
where we take $p(D_{\text{EM}}|D_{\text{GW}},H_0,I) = 1$, under the assumption that if the event was detected in gravitational waves, it will be detectable to EM telescopes.  While for current telecsopes and GW detection horizons, this is a reasonable assumption, it will have to be considered in more detail in the future.

In general, the assumption with the EM counterpart case is that observation of the counterpart will allow the identification of one (or more than one) of the galaxies in the neighboring region as the host of the GW event, and provide a redshift in this manner.

From here, the counterpart case can be broken down into two further methods: ``direct'' and ``pencil-beam''. 

\subsubsection{Direct}
This method assumes that the counterpart has been unambiguously linked to the host galaxy of the GW event, such that the redshift and sky location of that galaxy can be taken to be the redshift and sky location of the GW event with certainty.  In this case the likelihood from Eq. \ref{Eq:counterpart} can be calculated by evaluating the GW likelihood at the delta function location of the counterpart in $z$ and $\Omega$, and evaluating $p(D_{\text{GW}}|H_0,I)$ as $\iiint p(D_{\text{GW}}|H_0,z,\Omega,I) p(z)p(\Omega)p(M|H_0,I) dz d\Omega dM$ (note that this is independent of galaxy catalogue data).



\subsubsection{Pencil-beam}
The pencil-beam method makes the assumption that while the sky location of the galaxy associated with the counterpart is that of the GW event, the true host may be hidden behind that galaxy, and therefore there returns the question of whether the host is inside or outside the galaxy catalogue.  In this case, the likelihood takes the same form as in the statistical case, but evaluated along the line of sight of the counterpart\footnote{In future, this (and the ``direct'' method) should be expanded to cover a finite patch of sky, if more than one potential host galaxy (or none at all) could be associated with the counterpart.}.



\subsection{A brief note on luminosity weighting}
Whenever there is ambiguity about the host of a GW event, the possibility of luminosity weighting must be considered.

Included in the term $I$ on the right-hand side of all of these equations is one very important assumption: that there really was a GW source (here we shall denote it $s$), and the detection was not a false alarm.  In the majority of calculations, the presence of this term can be taken as granted and safely ignored, as it has no bearing on the result.  For example:
\begin{equation}
\begin{aligned}
p(z|s,I) &= \dfrac{p(s|z,I) p(z|I)}{p(s|I)} 
\\ &= \dfrac{p(s|I) p(z|I)}{p(s|I)} 
\\ &= p(z|I),
\end{aligned}
\end{equation}
as we do not modify our prior on $z$ because of our knowledge.

However, there is an important case in which this assumption is true only part of the time:
\begin{equation}
\begin{aligned}
p(M|s,H_0,I) &= \dfrac{p(s|M,H_0,I)p(M|H_0,I)}{p(s|H_0,I)} 
\\ &= p(M|H_0,I) 
\end{aligned}
\end{equation}
which is \emph{only} true if we believe that the probability of a particular galaxy being host to a gravitational wave event is independent of the galaxy's absolute magnitude. That is, when there is no luminosity weighting.

In general, we define $p(M|H_0,I)$ as a distribution of absolute magnitudes represented by the Schechter function, as we believe this mirrors the distribution of absolute magnitudes for all the galaxies in the universe.  If we believe that the probability of a given galaxy being host to the gravitational wave source is dependent on the galaxy's absolute magnitude, then $p(s|M,H_0,I)$ takes some non-constant value.  For example:
\begin{equation}
\begin{aligned}
p(s|M,H_0,I) &\propto 
\begin{cases}
L(M(H_0)) & \text{if luminosity weighted}\\
\text{const} & \text{if unweighted}
\end{cases}
\end{aligned}
\end{equation}

As will be seen below, all $p(s|H_0,I)$ terms will cancel out in the numerator and denominator, and therefore can be safely ignored.






\section{Individual components of the statistical case \label{Sec: Components}}
Now to consider the individual components of Eq. \ref{Eq:sum G}.  Note that in the cases where the integration limits are not specified, they can be assumed to cover the full parameter space.


\subsection{Likelihood when host is in catalogue: $p(x_{\text{GW}}|H_0,G,D_{\text{GW}},I)$}


Marginalising over redshift, sky location, apparent magnitude and absolute magnitude:
\begin{equation}
\begin{aligned}
p(x_{\text{GW}}|H_0,G,D_{\text{GW}},I) &= \dfrac{\iiiint p(x_{\text{GW}}|H_0,G,z,\Omega,m,M,I) p(z,\Omega,m,M|s,H_0,G,I) dz d\Omega dm dM}{\iiiint p(D_{\text{GW}}|H_0,G,z,\Omega,m,M,I) p(z,\Omega,m,M|s,H_0,G,I) dz d\Omega dm dM}
\\ &= \dfrac{\iiiint p(x_{\text{GW}}|H_0,z,\Omega,I) p(z,\Omega,m,M|s,H_0,G,I) dz d\Omega dm dM}{\iiiint p(D_{\text{GW}}|H_0,z,\Omega,I) p(z,\Omega,m,M|s,H_0,G,I) dz d\Omega dm dM}
\end{aligned}
\end{equation}
which is true \emph{if} we can assume that both $x_{\text{GW}}$ and $D_{\text{GW}}$ are independent of $G$, $m$ and $M$.  Here, the dependence of the priors on $G$ simply means that we take the prior to be the galaxies within the catalogue, as a series of delta functions with specific $z$, $\Omega$ and $m$ values.

As the priors on $z$, $\Omega$, $m$ and $M$ are connected through the specific galaxies inside the catalogue, expanding must be done with care:
\begin{equation}
\begin{aligned}
p(z,\Omega,m,M|s,H_0,G,I) &= p(M|s,H_0,z,\Omega,m,G,I)p(z,\Omega,m|s,H_0,G,I)
\\ &= \dfrac{p(s|M,H_0,z,\Omega,m,G,I) p(M|H_0,z,\Omega,m,G,I)}{p(s|H_0,z,\Omega,m,G,I)} \dfrac{p(s|H_0,z,\Omega,m,G,I) p(z,\Omega,m|H_0,G,I)}{p(s|H_0,G,I)} 
\\ &= \dfrac{p(s|M,I) \delta(M-M(H_0,z,m)) p(z,\Omega,m|G,I)}{p(s|H_0,G,I)}
\\ &= \dfrac{p(s|M(H_0,z,m),I) p(z,\Omega,m|G,I)}{p(s|H_0,G,I)}
\end{aligned}
\end{equation}


Substituting this back in:
\begin{equation}
\begin{aligned}
p(x_{\text{GW}}|H_0,G,D_{\text{GW}},I) &= \dfrac{p(s|H_0,G,I)}{p(s|H_0,G,I)} \dfrac{\iiint p(x_{\text{GW}}|H_0,z,\Omega,I) p(s|M(H_0,z,m),I) p(z,\Omega,m|G,I) dz d\Omega dm}{\iiint p(D_{\text{GW}}|H_0,z,\Omega,I) p(s|M(H_0,z,m),I) p(z,\Omega,m|G,I) dz d\Omega dm}
\\ &= \dfrac{\iiint p(x_{\text{GW}}|H_0,z,\Omega,I) p(s|M(H_0,z,m),I) p(z,\Omega,m|G,I) dz d\Omega dm}{\iiint p(D_{\text{GW}}|H_0,z,\Omega,I) p(s|M(H_0,z,m),I) p(z,\Omega,m|G,I) dz d\Omega dm}
\\ &= \dfrac{\sum^N_{i=1} p(x_{\text{GW}}|H_0,z_i,\Omega_i,I) p(s|M(H_0,z_i,m_i),I)}{\sum^N_{i=1} p(D_{\text{GW}}|H_0,z_i,\Omega_i,I) p(s|M(H_0,z_i,m_i),I)}
\end{aligned}
\end{equation}

In the unweighted case, this simplifies to the following:
\begin{equation}
\begin{aligned}
p(x_{\text{GW}}|H_0,G,D_{\text{GW}},I) &= \dfrac{\sum^N_{i=1} p(x_{\text{GW}}|H_0,z_i,\Omega_i,I) }{\sum^N_{i=1} p(D_{\text{GW}}|H_0,z_i,\Omega_i,I)}
\end{aligned}
\end{equation}



\subsection{Probability the host galaxy is in the galaxy catalogue: $p(G|H_0,D_{\text{GW}},I)$}


\begin{equation}
\begin{aligned}
p(G|H_0,D_{\text{GW}},I) &= \iiiint p(G,z,\Omega,m,M|H_0,D_{\text{GW}},I) dz d\Omega dm dM
\\ &= \iiiint p(G|H_0,D_{\text{GW}},z,\Omega,m,M,I) p(z,\Omega,m,M|H_0,D_{\text{GW}},I) dz d\Omega dm dM
\\ &= \iiiint \Theta[m_{\text{th}}-m] \dfrac{p(D_{\text{GW}}|H_0,z,\Omega,m,M,I) p(z,\Omega,m,M|H_0,I)}{p(D_{\text{GW}}|H_0,I)}  dz d\Omega dm dM
\\ &=  \dfrac{\iiiint \Theta[m_{\text{th}}-m] p(D_{\text{GW}}|H_0,z,\Omega,m,M,I) p(z,\Omega,m,M|H_0,I) dz d\Omega dm dM}{\iiiint p(D_{\text{GW}},z,\Omega,m,M|H_0,I) dz d\Omega dm dM} 
\\ &=  \dfrac{\iiiint \Theta[m_{\text{th}}-m] p(D_{\text{GW}}|H_0,z,\Omega,I) p(z,\Omega,m,M|H_0,I) dz d\Omega dm dM}{\iiiint p(D_{\text{GW}}|H_0,z,\Omega,I) p(z,\Omega,m,M|H_0,I) dz d\Omega dm dM} 
\end{aligned}
\end{equation}
where $p(G|H_0,D_{\text{GW}},z,\Omega,m,M,I)$ is a heaviside step function around $m = m_{\text{th}}$: the apparent magnitude threshold of the galaxy catalogue\footnote{Here $m_{\text{th}}$ is assumed to be uniform across the sky.  In the future this assumption will be replaced by an $m_{\text{th}}$ which is allowed to vary over the sky.}.

\begin{equation}
\begin{aligned}
p(G|H_0,D_{\text{GW}},I) &= \dfrac{\iiiint \Theta[m_{\text{th}}-m] p(D_{\text{GW}}|H_0,z,\Omega,I) p(z)p(\Omega)p(M|H_0,I)\delta(m - m(z,H_0,M)) dz d\Omega dm dM}{\iiiint p(D_{\text{GW}}|H_0,z,\Omega,I) p(z)p(\Omega)p(M|H_0,I)\delta(m - m(z,H_0,M)) dz d\Omega dm dM}
\\ &= \dfrac{\iiint \Theta[m_{\text{th}}-m(z,H_0,M)] p(D_{\text{GW}}|H_0,z,\Omega,I) p(z)p(\Omega)p(M|H_0,I)dz d\Omega dM}{\iiint p(D_{\text{GW}}|H_0,z,\Omega,I) p(z)p(\Omega)p(M|H_0,I) dz d\Omega dM}
\\ &= \dfrac{\int^{z(H_0,m_{\text{th}},M)}_0 dz \int d\Omega \int dM p(D_{\text{GW}}|H_0,z,\Omega,I) p(z)p(\Omega)p(M|H_0,I)}{\iiint p(D_{\text{GW}}|H_0,z,\Omega,I) p(z)p(\Omega)p(M|H_0,I) dz d\Omega dM}
\end{aligned}
\end{equation}


\subsubsection{The luminosity weighted case}
In the case where we consider luminosity weighting, this takes the following form:
\begin{equation}
\begin{aligned}
p(G|H_0,D_{\text{GW}},I) &= \dfrac{\int^{z(H_0,m_{\text{th}},M)}_0 dz \int d\Omega \int dM p(D_{\text{GW}}|H_0,z,\Omega,I) p(z)p(\Omega) p(s|M,I) p(M|H_0,I)}{\iiint p(D_{\text{GW}}|H_0,z,\Omega,I) p(z)p(\Omega) p(s|M,I) p(M|H_0,I) dz d\Omega dM}
\end{aligned}
\end{equation}


\subsection{Probability the host galaxy is not in the galaxy catalogue: $p(\bar{G}|H_0,D_{\text{GW}},I)$}

As the probabilities of being in the catalogue and not in the catalogue must add up the one, we can calculate the counterpart to $p(G|H_0,D_{\text{GW}},I)$ as follows:
\begin{equation}
\begin{aligned}
p(\bar{G}|H_0,D_{\text{GW}},I) &= 1 - p(G|H_0,D_{\text{GW}},I)
\end{aligned}
\end{equation}




\subsection{Likelihood when host is not in catalogue: $p(x_{\text{GW}}|H_0,\bar{G},D_{\text{GW}},I)$}
Similarly:
\begin{equation}
\begin{aligned}
p(x_{\text{GW}}|H_0,\bar{G},D_{\text{GW}},I) &= \dfrac{\iiiint p(x_{\text{GW}}|H_0,\bar{G},z,\Omega,I) p(z,\Omega,m,M|H_0,\bar{G},I) dz d\Omega dm dM}{\iiiint p(D_{\text{GW}}|H_0,\bar{G},z,\Omega,I) p(z,\Omega,m,M|H_0,\bar{G},I) dz d\Omega dm dM}
\\ &= \dfrac{\iiiint p(x_{\text{GW}}|H_0,z,\Omega,I) p(z,\Omega,m,M|H_0,\bar{G},I) dz d\Omega dm dM}{\iiiint p(D_{\text{GW}}|H_0,z,\Omega,I) p(z,\Omega,m,M|H_0,\bar{G},I) dz d\Omega dm dM}
\\ &= \dfrac{\iiiint p(x_{\text{GW}}|H_0,z,\Omega,I) \dfrac{p(\bar{G}|H_0,z,\Omega,m,M,I)p(z,\Omega,m,M|H_0,I)}{p(\bar{G}|H_0,I)} dz d\Omega dm dM}{\iiiint p(D_{\text{GW}}|H_0,z,\Omega,I) \dfrac{p(\bar{G}|H_0,z,\Omega,m,M,I)p(z,\Omega,m,M|H_0,I)}{p(\bar{G}|H_0,I)} dz d\Omega dm dM}
\\ &= \dfrac{p(\bar{G}|H_0,I)}{p(\bar{G}|H_0,I)}\dfrac{\iiiint p(x_{\text{GW}}|H_0,z,\Omega,I) p(\bar{G}|m,I)p(z,\Omega,m,M|H_0,I) dz d\Omega dm dM}{\iiiint p(D_{\text{GW}}|H_0,z,\Omega,I) p(\bar{G}|m,I)p(z,\Omega,m,M|H_0,I) dz d\Omega dm dM}
\\ &= \dfrac{\iiiint p(x_{\text{GW}}|H_0,z,\Omega,I) p(\bar{G}|m,I)p(z,\Omega,m,M|H_0,I) dz d\Omega dm dM}{\iiiint p(D_{\text{GW}}|H_0,z,\Omega,I) p(\bar{G}|m,I)p(z,\Omega,m,M|H_0,I) dz d\Omega dm dM}
\\ &= \dfrac{\iiiint p(x_{\text{GW}}|H_0,z,\Omega,I) \Theta(m-m_{\text{th}})p(z)p(\Omega)\delta(m-m(z,H_0,M))p(M|H_0,I) dz d\Omega dm dM}{\iiiint p(D_{\text{GW}}|H_0,z,\Omega,I) \Theta(m-m_{\text{th}})p(z)p(\Omega)\delta(m-m(z,H_0,M))p(M|H_0,I) dz d\Omega dm dM}
\\ &= \dfrac{\iiint p(x_{\text{GW}}|H_0,z,\Omega,I) \Theta(m(z,H_0,M)-m_{\text{th}}) p(z)p(\Omega)p(M|H_0,I) dz d\Omega dM}{\iiint p(D_{\text{GW}}|H_0,z,\Omega,I) \Theta(m(z,H_0,M)-m_{\text{th}})p(z)p(\Omega)p(M|H_0,I) dz d\Omega dM}
\\ &= \dfrac{\int^\infty_{z(H_0,m_{\text{th}},M)} \iint p(x_{\text{GW}}|H_0,z,\Omega,I) p(z)p(\Omega)p(M|H_0,I) dz d\Omega dM}{\int^\infty_{z(H_0,m_{\text{th}},M)} \iint p(D_{\text{GW}}|H_0,z,\Omega,I) p(z)p(\Omega)p(M|H_0,I) dz d\Omega dM}
\end{aligned}
\end{equation}


\subsubsection{The luminosity weighted case}
In the case where we consider luminosity weighting, this takes the following form:
\begin{equation}
\begin{aligned}
p(x_{\text{GW}}|H_0,\bar{G},D_{\text{GW}},I) &= \dfrac{\int^\infty_{z(H_0,m_{\text{th}},M)} \iint p(x_{\text{GW}}|H_0,z,\Omega,I) p(z) p(\Omega) p(s|M,I) p(M|H_0,I) dz d\Omega dM}{\int^\infty_{z(H_0,m_{\text{th}},M)} \iint p(D_{\text{GW}}|H_0,z,\Omega,I) p(z) p(\Omega) p(s|M,I) p(M|H_0,I) dz d\Omega dM}
\end{aligned}
\end{equation}





\section{Converting from maths to code \label{Sec: maths2code}}
Below are a few differences to bear in mind between the maths presented in section \ref{Sec: Components} and the functions as they appear in gwcosmo.



\subsection{Separating distance and sky location for GW events}
Mathematically speaking, the GW likelihood is conditioned jointly on $d_L$ and $\Omega$, and should not be separated:
\begin{equation}
\begin{aligned}
p(x_{\text{GW}}|d_L,\Omega) &\neq p(x_{\text{GW}}|d_L) p(x_{\text{GW}}|\Omega).
\end{aligned}
\end{equation}
However, in order to reduce the number of integrals required to calculate $p(x|H_0,I)$ we make this approximation, and calculate the integral over sky separately from the ones over redshift.


Also, we bear in mind that when we create a KDE over the event's posterior samples, it will include whichever prior was chosen in order to generate the samples, which must be removed if a different prior is to be applied.  Specifically, this means that the $d_L^2$ prior which went into the samples needs to be removed so that it can be replaced with the uniform in comoving volume prior, $p(z|I)$.  If skymaps are being used to provide sky localisation information, the prior on sky has already been removed.



\subsection{Sky-averaged $p(D|H_0,I)$}
While in reality the probability of detecting a GW event varies across the sky due to antenna patterns, in the code we marginalise over the sky location in order to reduce the number of integrals required to calculate $p(D|H_0,I)$.  This is similar (but not identical) to marginalising over time of detection due to the rotation of the earth.  Ideally in the future, neither of these approximations will be used.




\section{For the future: the pixel-based method \label{Sec: Future}}
In order to take into account the fact that galaxy catalogues have varying levels of completeness across the sky, we consider a method in which the sky is gridded up into equally-sized pieces, which are later summed.

\begin{equation}
\begin{aligned}
p(x_{\text{GW}}|H_0,D_{\text{GW}},I) &= \int p(x_{\text{GW}},\Omega|H_0,D_{\text{GW}},I) d\Omega
\\ & = \int p(x_{\text{GW}}|H_0,D_{\text{GW}},\Omega,I) p(\Omega|H_0,D_{\text{GW}},I) d\Omega
\\ & = \int p(x_{\text{GW}}|H_0,D_{\text{GW}},\Omega,I) \dfrac{p(D_{\text{GW}}|H_0,\Omega,I)p(\Omega|H_0,I)}{p(D_{\text{GW}}|H_0,I)}  d\Omega
\\ &= \dfrac{1}{p(D_{\text{GW}}|H_0,I)} \int p(x_{\text{GW}}|H_0,D_{\text{GW}},\Omega,I) p(D_{\text{GW}}|H_0,\Omega,I)p(\Omega|I) d\Omega
\\ &= \dfrac{1}{p(D_{\text{GW}}|H_0,I)} \sum^{N_{\text{pix}}}_i \bigg[p(x_{\text{GW}}|H_0,D_{\text{GW}},\Omega_i,I) p(D_{\text{GW}}|H_0,\Omega_i,I)p(\Omega_i|I)\bigg]
\end{aligned} 
\end{equation}

Looking specifically at $p(x_{\text{GW}}|H_0,D_{\text{GW}},\Omega_i,I)$ and expanding as in previous sections:
\begin{equation}
\begin{aligned}
p(x_{\text{GW}}|H_0,D_{\text{GW}},\Omega_i,I) &= \dfrac{p(x_{\text{GW}}|H_0,G,\Omega_i,I)}{p(D_{\text{GW}}|H_0,G,\Omega_i,I)} p(G|H_0,D_{\text{GW}},\Omega_i,I) + \dfrac{p(x_{\text{GW}}|H_0,\bar{G},\Omega_i,I)}{p(D_{\text{GW}}|H_0,\bar{G},\Omega_i,I)} p(\bar{G}|H_0,D_{\text{GW}},\Omega_i,I),
\end{aligned}
\end{equation}
and so the final expression becomes:
\begin{equation}
\begin{aligned}
p(x_{\text{GW}}|H_0,D_{\text{GW}},I) = \dfrac{1}{p(D_{\text{GW}}|H_0,I)} &\sum^{N_{\text{pix}}}_i \Bigg[ \bigg( \dfrac{p(x_{\text{GW}}|H_0,G,\Omega_i,I)}{p(D_{\text{GW}}|H_0,G,\Omega_i,I)} p(G|H_0,D_{\text{GW}},\Omega_i,I) \\ &+ \dfrac{p(x_{\text{GW}}|H_0,\bar{G},\Omega_i,I)}{p(D_{\text{GW}}|H_0,\bar{G},\Omega_i,I)} p(\bar{G}|H_0,D_{\text{GW}},\Omega_i,I) \bigg) \times p(D_{\text{GW}}|H_0,\Omega_i,I)p(\Omega_i|I) \Bigg]
\end{aligned} 
\end{equation}

It is also worth noting that, for a suitably fine grid, the choice of 3D vs 2+1D for dealing with GW data is removed, as for every position in the sky the corresponding distance posterior is used, and so this method is inherently ``3D''.





\end{document}